\newcommand{\nm[1]}{\{#1\}}
\begin{proof}[Beweis]
	\begin{math}
		Beh:
	\end{math}
	\begin{align*}
		&K_1=\[\lambda\]=\{w\in\{0,1\}^*\mid\text{ w ist als Binärzahl betrachtet durch 3 teilbar mit Rest 0}\}\\
		&K_2=\[1\]=\{w\in\{0,1\}^*\mid\text{ w ist als Binärzahl betrachtet durch 3 teilbar mit Rest 1}\}\\
		&K_3=\[10\]=\{w\in\{0,1\}^*\mid\text{ w ist als Binärzahl betrachtet durch 3 teilbar mit Rest 2}\}
	\end{align*}
	\begin{proof}[$K_1\cup K_2\cup K_3=\{a,b\}^*,K_i\;sind\;paarweise\;disjunkt$]
	\end{proof}
\end{proof}

\begin{center}
	$M=(\{0,1\},\{\equiv_3=0,\equiv_3=1,\equiv_3=2\},\delta,\{\equiv_3=0\},\{\equiv_3=0\})$\\
	\begin{tikzpicture}[shorten >=1pt,node distance=3cm,on grid,>={Stealth[round]},thick]
		
		\node (q0) [state,initial, accepting, initial text = {}] {$\equiv_3=0$};
		\node (q1) [state, right of = q0] {$\equiv_3=1$};
		\node (q2) [state, right of = q1] {$\equiv_3=2$};
		
		\path [-stealth, thick]
		(q0) edge [loop above] node [above] {$0$} (q0)
		(q0) edge [bend left] node [above] {$1$} (q1)
		(q1) edge [bend left] node [above] {$1$} (q0)
		(q1) edge [bend left] node [above] {$0$} (q2)
		(q2) edge [bend left] node [above] {$0$} (q1)
		(q2) edge [loop above] node [above] {$1$} (q2);

	\end{tikzpicture}
	
	$\delta:$
	\begin{table}[h]
		\centering
		\begin{tabular}{|c|c|c|}
			\hline
			Zustand & $0$ & $1$ \\
			\hline\hline
			$\emptyset$&$\emptyset$&$\emptyset$\\
			\hline
			$\equiv_3=0$&\nm[$\equiv_3=0$]&\nm[$\equiv_3=1$]\\
			\hline
			$\equiv_3=1$&\nm[$\equiv_3=2$]&\nm[$\equiv_3=0$]\\
			\hline
			$\equiv_3=1$&\nm[$\equiv_3=1]$&\nm[$\equiv_3=2$]\\
			\hline
		\end{tabular}
	\end{table}
\end{center}
Wenn eine Zahl durch 3 Teilbar ist gibt es nur drei Möglichkeiten. Der Rest kann 0,1 oder 2 sein. Im DFA $M$ sind diese drei Zustände angegeben und in $\delta$ ihre Übergänge definiert. Man betrachtet die Wirkung der einzelnen Buchstaben auf das gesamte Wort und erkennt: wenn man in $\equiv_3=0$ ist ändert eine 0 nichts an der Teilbarkeit durch 3. Eine eins hingegen ehröht den Rest bei Division durch 3 auf 1. Deshalb landet man in $\equiv_3=1$. Wenn nun direkt eine 1 kommt wandert man wieder zurück, da dann $11_{bin}=3_{dez}$ angehängt wurde, was durch 3 mit Rest 0 teilbar ist. Fügt man hingegen eine 0 an und hat das Teilwort $10$ angehängt, so landet man in $\equiv_3=2$, da $10_{bin}=2_{dez}$. Von hier aus kann man mit einer 0 zurück, da $101_{bin}=4_dez\equiv_3=1$. Sollte man eine 1 lesen bleibt man solange im Zustand bis eine 0 kommt. Dies liegt daran, dass $1011_{bin}=11_{dez}\equiv_3=2,10111_{bin}=23\equiv_3=2,...$. $\Rightarrow G\in REG$.
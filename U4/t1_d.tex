\newcommand{\nm[1]}{\{#1\}}
\begin{proof}[Jedes Wort gerader Länge ist in D]
	Wenn $w\in\{a,b\}^*$ gerade ist, dann gibt es eine Zahl $k\in\mathbb{N}$, so dass $\mid w\mid=2k\Rightarrow w$ kann in zwei Teile der Länge $k$ aufgeteilt werden: $x,y\in\{a,b\}^*:\mid x\mid=\mid y\mid=k\Rightarrow w\in D$
\end{proof}
\begin{proof}[Jedes Wort in L ist gerader Länge]
	Für jedes Wort in $D$ mit gerader Länge gilt: $Sei\;w\in D$ nach Definition von $D$ folgt: $\exists x,y\in\{a,b\}^*:\mid x\mid=\mid y\mid$ und $w=xy\Rightarrow\mid w\mid=\mid x\mid+\mid y\mid=2\mid x\mid\Rightarrow w$ hat gerade Länge.
\end{proof}
\noindent
$D\in REG.$
\begin{center}
	$M=(\{a,b\},\{q_0,q_1\},\delta,\{q_0\},\{q_0\})$\\
	\begin{tikzpicture}[shorten >=1pt,node distance=3cm,on grid,>={Stealth[round]},thick]
		
		\node (q0) [state,initial, accepting, initial text = {}] {$q_0$};
		\node (q1) [state, right of = q0] {$q_1$};
		
		\path [-stealth, thick]
		(q0) edge [bend left] node [above] {$a,b$} (q1)
		(q1) edge [bend left] node [above] {$a,b$} (q0);
	\end{tikzpicture}

	$\delta:$
	\begin{table}[h]
		\centering
		\begin{tabular}{|c|c|c|}
			\hline
			Zustand & $a$ & $b$ \\
			\hline\hline
			$\emptyset$&$\emptyset$&$\emptyset$\\
			\hline
			$q_0$&\nm[$q_1$]&\nm[$q_1$]\\
			\hline
			$q_1$&\nm[$q_0$]&\nm[$q_0$]\\
			\hline
		\end{tabular}
	\end{table}
\end{center}
\qed